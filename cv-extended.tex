%!TEX TS-program = xelatex
%!TEX encoding = UTF-8 Unicode

%-------------------------------------------------------------------------------
% CONFIGURATIONS
%-------------------------------------------------------------------------------
% A4 paper size by default, use 'letterpaper' for US letter
\documentclass[11pt, a4paper]{preamble/awesome-cv-novo}

\hypersetup{
  pdftitle={Curriculum Vitae},
  pdfauthor={Lento Manickathan},
  pdfsubject={Curriculum Vitae},
  pdfkeywords={CV, Aerospace Engineer, Lento, Manickathan}
}

% Configure page margins with geometry
\geometry{left=2cm, top=2cm, right=2cm, bottom=2cm, footskip=.5cm}

%===============================================================================
%   PERSONAL INFORMATION
%===============================================================================

% Available options: circle|rectangle,edge/noedge,left/right
% \photo[rectangle,edge,right]{./examples/profile}
\photo[circle,edge,right]{profile.png}
\name{Lento}{Manickathan}
\nametitle{Ph.D.}
\position{Aerospace Engineer{\enskip\cdotp\enskip}Postdoctoral Researcher}
\address{Geissbergstrasse 22, 8200 Schaffhausen, Switzerland}

% \mobile{(+41) xx xxx xxxx}
\email{lento.manickathan@gmail.com}
\homepage{lento234.ch}
\github{lento234}
% \linkedin{lento-manickathan}
% \gitlab{gitlab-id}
% \stackoverflow{SO-id}{SO-name}
% \twitter{@mrlento234}
% \skype{skype-id}
% \reddit{reddit-id}
% \medium{madium-id}
% \googlescholar{b8RcAAAAJ}{}
%% \firstname and \lastname will be used
\extrainfo{Ausweis C}

% \quote{``There is nothing more permanent than a temporary solution."}


%-------------------------------------------------------------------------------
\begin{document}

% Print the header with above personal informations
% Give optional argument to change alignment(C: center, L: left, R: right)
\makecvheader[L]

% Print the footer with 3 arguments(<left>, <center>, <right>)
% Leave any of these blank if they are not needed
\makecvfooter
  % {\today}
  % {Lento Manickathan}
  {Lento Manickathan\enskip\textbar\enskip CV}
  {}
  {\thepage}


%===========================================================================
%   SUMMARY
%===========================================================================

\cvsection{Summary}

\begin{cvparagraph}

Aerospace Engineer with interests in high-performance computing, machine learning and computer vision techniques. A brief summary of my expertise:

\begin{cventries}
    \begin{cvitems}
      \item {HPC programming in Python.}
      \item {Machine learning with PyTorch.}
      \item {CFD simulation with OpenFOAM.}
      \item {PIV experimental skills.}
      \item {Cloud computing, GPU computing.}
      \item {Administration of HPC system.}
    \end{cvitems}
\end{cventries}

\end{cvparagraph}

\vspace{4mm}

%===========================================================================
%   ACADEMIC & PROFESSIONAL EXPERIENCE
%===========================================================================

\cvsection{Academic \& Professional Experience}

\begin{cventries}

  \cventry
    {Postdoctoral researcher} % Job title
    {Empa (Swiss Federal Laboratories for Materials Science and Technology)} % Organization
    {Z\"urich, Switzerland} % Location
    {Jun 2019 - Present} % Date(s)
    {
      \begin{cvitems} % Description(s) of tasks/responsibilities
        \item {Application of machine learning in quantitative flow visualization.}
        \item {Supervised optical flow algorithms for PIV.}
        \item {Convolutional Neural Networks (CNN) in PyTorch.}
        \item {Additional responsibilities: Lab HPC support, Fluid Tunnel support, Deputy Laser safety officer (LSO), and Deputy data management.}
      \end{cvitems}
    }

  \cventry
    {Scientific Assistant}
    {ETH Zurich}
    {Z\"urich, Switzerland}
    {May 2015 - May 2019}
    {
      \begin{cvitems}
        \item {Numerical and experimental research at Empa.}
        \item {Neutron radiography at Paul Scherrer Institute (PSI).}
        \item {Teaching assistant for Application of CFD in buildings.}
        \item {Supervision of master thesis project: \textit{Praharsh Pai Raikar}.}
      \end{cvitems}
    }

    \cventry
    {Research Intern}
    {Shell Global Solutions}
    {Rijswijk, The Netherlands}
    {Sep 2012 - Feb 2013}
    {
      \begin{cvitems}
        \item {Investigating the combustion of hydrogen-rich Syngas fuel in gas turbine.}
      \end{cvitems}
    }

\end{cventries}


%===========================================================================
%   EDUCATION
%===========================================================================

\cvsection{Education}

\begin{cventries}

  \cventry
    {Ph.D. in Mechanical Engineering} % Degree
    {ETH Zurich} % Institution
    {Z\"urich, Switzerland} % Location
    {May 2015 - Jun. 2019} % Date(s)
    {
      \begin{cvitems} % Description(s) bullet points
        \item {\textbf{Thesis}: Impact of Vegetation on Urban Microclimate.}
        \item {\textbf{Advisor}: Prof. Dr. Jan Carmeliet}
        \item {Development of a coupled soil-vegetation-air-radiation model in C++ within the OpenFOAM library.}
        \item {Wind tunnel study of flow past model and natural plants using PIV.}
        \item {X-ray tomography of small natural plants and high-performance big data analysis in python (Dask, HDF5, Numba, Scikit-image).}
      \end{cvitems}
    }

  \cventry
    {M.Sc. in Aerospace Engineering}
    {TU Delft (Delft University of Technology)}
    {Delft, The Netherlands}
    {Sep 2011 - Dec 2014}
    {
      \begin{cvitems}
        \item {\textbf{Major}: Aerodynamics and Wind Engineering.}
        \item {\textbf{Thesis}: Hybrid Eulerian-Lagrangian Vortex Particle Method: Developing a fast and accurate numerical method for the application of Vertical-Axis Wind Turbine (VAWT).}
        \item {\textbf{Advisor}: Prof. dr. ir. C.J. (Carlos) Simão Ferreira}
        \item {Development of a high-performance numerical method in python with Cython and GPU (CUDA) acceleration.}
      \end{cvitems}
    }

  \cventry
    {B.Sc. in Aerospace Engineering}
    {TU Delft (Delft University of Technology)}
    {Delft, The Netherlands}
    {Sep 2008 - Aug 2011}
    {
      \begin{cvitems}
        \item {\textbf{Minor}: Wind Energy and Sustainability.}
        \item {\textbf{Thesis}: Designing a multi-purpose autonomous aerial monitoring aircraft.}
        \item {Design a UAV that can cope with severe weather conditions while performing a variety of sensing and monitoring tasks.}
      \end{cvitems}
    }

\end{cventries}


%===========================================================================
%   EXTRACURRICULAR ACTIVITIES
%===========================================================================

\cvsection{Extracurricular Activities}

\begin{cventries}

  \cventry
    {Editor} % Degree
    {Leonardo Times Magazine} % Institution
    {Delft, The Netherlands} % Location
    {Sep 2011 - Aug 2012} % Date(s)
    {
      \begin{cvitems} % Description(s) bullet points
        \item {Journal of the Society for Aerospace Engineering students, the VSV \textit{Leonardo da Vinci} at the TU Delft.}
        \item {In charge of \textit{Current Affairs} section.}
      \end{cvitems}
    }

  \cventry
    {Powertrain Engineer}
    {TU Delft Formula Student}
    {Delft, The Netherlands}
    {Sep 2009 - Jul 2010}
    {
      \begin{cvitems}
        \item {In charge of designing the powertrain intake system.}
        \item {Design and production of the carbon-fibre intake system.}
        \item {\textit{2010 Formula Student Germany Champion.}}
      \end{cvitems}
    }

\end{cventries}


%===========================================================================
%   COMPUTER SKILLS
%===========================================================================

\cvsection{Skills}

\cvsubsubsection{Scientific Programming}

  \begin{cvskills}
    \cvskill
      {CAD}
      {Blender \sep CATIA}
    \cvskill
      {CFD}
      {FEniCS \sep Fluent \sep OpenFOAM}
    \cvskill
      {Programming}
      {C++ \sep MATLAB \sep Python \sep R \sep Shell}
    \cvskill
      {Python Libraries (HPC)}
      {CuPy \sep Cython \sep Dask \sep H5py \sep MPI4py \sep Numba \sep NumPy \sep Pandas \sep SciPy}
    \cvskill
      {Python Libraries (ML)}
      {PyTorch \sep Scikit-learn}
    \cvskill
      {Python Libraries (Plotting)}
      {Dash \sep Matplotlib \sep Scikit-image}
  \end{cvskills}

\vspace{3mm}
\cvsubsubsection{Software Development}

  \begin{cvskills}

    \cvskill
      {Automation}
      {Ansible}
    \cvskill
      {CI / CD}
      {Git (GitHub, Gitlab) \sep Travis CI}
    \cvskill
      {Cloud}
      {Amazon AWS (EC2)}
    \cvskill
      {Container}
      {Docker \sep Kubernetes \sep Sarus \sep Vagrant}
    \cvskill
      {Database}
      {InfluxDB \sep MariaDB}
    \cvskill
      {Embedded}
      {Arduino \sep Raspberry Pi \sep NVIDIA Jetson Nano}
    \cvskill
      {HPC}
      {SLURM}
    \cvskill
      {Markup / Typesetting}
      {Jinja \sep LaTeX \sep Markdown \sep MkDocs \sep Vim}
    \cvskill
      {ML Libraries}
      {PyTorch \sep Scikit-learn}
    \cvskill
      {OS}
      {Linux (Debian, Red Hat) \sep MacOS \sep Windows}
    \cvskill
      {Web}
      {CSS \sep HTML5 \sep Nginx}
  \end{cvskills}

%===========================================================================
%   LANGUAGES
%===========================================================================

\cvsection{Languages}

  \begin{cvlanguages}
      \cvlanguage
      {\textbf{English} (Fluent), \textbf{German} (Conversational), \textbf{Malayalam} (Fluent), \textbf{Dutch} (Basic)}
  \end{cvlanguages}

%===========================================================================
%   HONORS & AWARDS
%===========================================================================

\cvsection{Honors \& Awards}

\begin{cvhonors}

  \cvhonor
    {Outstanding Oral Presentation} % Award
    {13\textsuperscript{th} Symposium on Urban Environment} % Event
    {Seattle, USA} % Location
    {2017} % Date(s)

    \cvhonor
    {Young Best Researcher} % Award
    {4\textsuperscript{th} International Conference on Countermeasures to Urban Heat Island} % Event
    {Singapore} % Location
    {2016} % Date(s)
\end{cvhonors}

%===========================================================================
%   PUBLICATIONS
%===========================================================================

\newpage

\cvsection{Publications}

\begin{cventries}

  \cvpubentry
  {Journals} % Title
  {
    \begin{cvitems}
      \item {\textbf{Manickathan, L.}, Mucignat, C., Lunati, I. (2022). Random displacement training for fluid flow
             motion estimation. {\bodyfontlight(\textit{in preparation})}}
      \item {\textbf{Manickathan, L.}, Defraeye, T., Carl, S., Richter, H., Allegrini, J., Derome, D., \& Carmeliet,
             J. (2022). A study on diurnal microclimate hysteresis and plant morphology of a \textit{Buxus
             sempervirens} using PIV, infrared thermography, and X-ray imaging. {\bodyfontlight\textit{Agricultural
             and Forest Meteorology 313}, 108722.}}
      \item {\textbf{Manickathan, L.}, Defraeye, T., Allegrini, J., Derome, D., \& Carmeliet, J. (2018). Parametric
             study of the influence of environmental factors and tree properties on the transpirative cooling
             effect of trees. {\bodyfontlight\textit{Agricultural and Forest Meteorology, 248}, 259-274.}}
      \item {\textbf{Manickathan, L.}, Defraeye, T., Allegrini, J., Derome, D., \& Carmeliet, J. (2018).
             Comparative study of flow field and drag coefficient of model and small natural trees in a wind tunnel.
             {\bodyfontlight\textit{Urban Forestry \& Urban Greening, 35}, 230–239.}}
    \end{cvitems}
  }

  \cvpubentry
  {Preprints}
  {
    \begin{cvitems}
      \item {Palha, A., \textbf{Manickathan, L.}, Ferreira, C. S., \& van Bussel, G. (2015). A hybrid
             Eulerian-Lagrangian flow solver. {\bodyfontlight\textit{arXiv preprint arXiv:1505.03368.}}}
    \end{cvitems}
  }

  \cvpubentry
  {Conferences}
  {
    \begin{cvitems}
      \item {\textbf{Manickathan, L.}, Kubilay, A., Defraeye, T., Allegrini, J., Derome, D., \& Carmeliet, J.
             Integrated CFD vegetation model with soil-plant-air water dynamics for studying the cooling potential
             of vegetation in an urban street canyon. {\bodyfontlight\textit{10th International Conference on
             Urban Climate/14th Symposium on the Urban Environment}, New York, NY, USA, 6 - 10 August 2018.}}
      \item {\textbf{Manickathan, L.}, Defraeye, T., Allegrini, J., Derome D., \& Carmeliet, J. Conjugate
             Vegetation Model for Evaluating Evapotranspirative Cooling in Urban Environment.
             {\bodyfontlight\textit{97th AMS Annual Meeting}, Seattle, WA, USA, 2017.}}
      \item {\textbf{Manickathan, L.}, Defraeye, T., Allegrini, J., Derome, D., \& Carmeliet, J. Aerodynamic
             characterization of model vegetation by wind tunnel experiments. {\bodyfontlight\textit{4th
             International Conference on Countermeasures to Urban Heat Island}, Singapore, 2016.}}
    \end{cvitems}
  }

\end{cventries}


\end{document}
